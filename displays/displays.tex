\chapter{Second Displays}
\label{ch:displays}

\begin{gram}[Purhasing Instructions]
\label{grm:displays::preamble}
If you are teaching in Fall 2020 and if you want to use a second
display for writing, then you can request a Wacom Pen Display by emailing 
\href
{mailto:jpacker@andrew.cmu.edu?subject=Fall 2020 Remote Equipment&body=Hi Jessica,}
{Jessica Packer by clicking here}.
%
\textbf{Please make sure that the ``Subject'' line of your email reads ``Fall 2020 Remote Equipment''}

If you do not know whether you will genuinely need this device, you can request a loaner to try it out.
%

Please let us know about your intentions by Friday (July 31).  If you don't know that you will need it for certain, you can give us an estimate, e.g., want it with 60\%, 90\% probability.  
%
These displays are on high demand and we would like to place an initial order as soon as possible.

\end{gram}


\section{Wacom Pen Display}
\label{ch:displays::wacom}

For effective remote lectures, having some display space for drawing and writing things can be very important.  When lecturing in a classroom, the lecturer typically has access to a white or a chalk  board for this purpose.  
%
Tablet computers such as iPad and Surface work reasonably well in many cases but in some lectures, a bigger space may be helpful and even necessary. 
%
We recommend using a Wacom Pen Display for this purpose.

\begin{gram}[16 inch Display]
\label{grm:displays::wacom::16}
The 16 inch ``Wacom Pen Display'' offers writeable display.
%
It can be attached as a second display to  Mac, Windows, and Linux computers.  
%
It does not have its own operating systems and is basically plug
and play after you set them up.
%
These displays use a ``passive pen'' that does not require charging.  

The key advantage of these displays over tablets is that they offer a fairly large amount of writing surface that can be used as an ``electronic whiteboard''.  
%
Another advantage is their lack of resident software and therefore their relatively simple use. 
%
One disadvantage is that they occupy a bit of space on your desk and they have multiple cables that must be plugged into a computer (possibly via an adapter).  
%
They are therefore best with a stationary setup, e.g., as a display attached to laptop or a desktop computer with its own dedicated display.
%
For this reason, they are probably most useful if you plan to do a fair amount of writing when teaching.

\textbf{More details.}

You can find full technical details on this display
\href
{https://estore.wacom.com/en-US/dtk-1660e-fhd-interactive-pen-display-us-dtk1660ek0a.html?country_code=US}
{on Wacom's web site}.

\end{gram}

%% \begin{gram}[Setup for Mac]
%% \label{grm:displays::wacom::16::mac-setup}

%% TODO: ENTER INSTRUCTIONS FOR SETTING UP ON MAC
%% \end{gram}


%% \begin{gram}[Setup for WINDOWS]
%% \label{grm:disys::wacom::16::win-setup}

%% TODO: ENTER INSTRUCTIONS FOR TING UP ON WINDOWS
%% \end{gram}

%% \subsection{Ap iPad (w/Apple Pencil}
%% \label{sec:displays::idap}

%% \begiram}[Apple iPad]
%% \label{grm:displays::ipad}
%% A new Apple iPad with Apple Pencil can be used as a drawing tablet for digital whiteboard content. \end{gram}
%%  \begin{gram}[iPad w/Zoom] You can connect your iPad to Zoom using the Zoom App for iPad and calling into the meeting. 
%%  From there, you can \textbf{Share Content} on the iPad and choose the \textbf{Whiteboard}. This will give you access to the Zoom whiteboard feature.

%% This video show how to use the Zoom whiteboard from the iPad app.

%% \video{https://www.youtube.com/embed/SCr010CMRsg}{How to Present a Remote Whiteboard Lecture with Zoom, iPad, and a digital pencil}
%% \end{gram}


%% \begin{gram}
%% You can also share your iPad screen and use another drawing app (ex. ProCreate or Adobe Sketchbook). For this, select \textbf{share screen} when you are 


%% \video{https://www.youtube.com/embed/6uO2F1BIM5M}{How to use Zoom Screen Sharing from an iPad}
%% \end{gram}


