\chapter{Second Displays}
\label{ch:displays}

\begin{gram}
\label{grm:displays::preamble}
TODO: PURCHASING INSTRUCTIONS
\href{mailto:jpacker@andrew.cmu.edu}{Jessica Packer}.
%
\end{gram}


\section{Pen Display}

For effective remote lectures, having some display space for drawing and writing things can be very important.  When lecturing in a classroom, the lecturer typically has access to a white or a chalk  board for this purpose.  
%
Tablet computers such as iPad and Surface work reasonably well in many cases but in some lectures, a bigger space may be helpful and even necessary. 
%
We recommend using a Wacom Pen Display for this purpose.

\subsection{Wacom Pen Display}
\label{sec:displays::wacom}

\begin{gram}[16 inch Display]
\label{grm:displays::wacom::16}
The 16 inch ``Wacom Pen Display'' offers writeable display.
%
It can be attached as a second display to  Mac, Windows, and Linux computers.  
%
It does not have its own operating systems and is basically plug
and play after you set them up.
%
These displays use a ``passive pen'' that does not require charging.  

\textbf{More details.}

You can find full technical details on this display
\href
{https://estore.wacom.com/en-US/dtk-1660e-fhd-interactive-pen-display-us-dtk1660ek0a.html?country_code=US}
{on Wacom's web site}

\end{gram}

\begin{gram}[Setup for Mac]
\label{grm:displays::wacom::16::mac-setup}

TODO: ENTER INSTRUCTIONS FOR SETTING UP ON MAC
\end{gram}


\begin{gram}[Setup for WINDOWN]
\label{grm:displays::wacom::16::win-setup}

TODO: ENTER INSTRUCTIONS FOR SETTING UP ON WINDOWS
\end{gram}


