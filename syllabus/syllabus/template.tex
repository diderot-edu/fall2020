\chapter{Syllabus and Policies}
\label{policies}
\begin{preamble}
\end{preamble}

\section{Overview}

\subsection{Synopsis}

\begin{gram}
\textbf{Lecture:}
X + Y at HH:MM - HH:MM pm

\textbf{Location:} 
[Remote, on Zoom] [Building and Room]

\textbf{Instructor:} 
\href{https://}{First Last Name} 

\textbf{Teaching Assistants:} ...

\textbf{Grading:} 
Grades will be available on Diderot.
\\
Midterms X\% (each), Final Y\%, Homework Z\%

\textbf{Textbooks:} 
...

\textbf{Announcements + Q\&A:} We will use Diderot for questions and any course announcements.

\textbf{Submitting Assignments:} 
Students will turn in their homework electronically using \href{https://www.gradescope.com}{Gradescope} [and Diderot].
\end{gram}


\begin{gram}[Course Description]
INSERT A BRIEF DESCRIPTION OF THE COURSE
\end{gram}
      
\begin{gram}[Course Prerequisites]
The prequisites for this course are:
          \begin{itemize}
            \item Course I
            \item Course II
            \end{itemize}

Please contact the instructors if you are unsure about your preparedness.
\end{gram}

\begin{gram}[Learning Goals]

\begin{itemize}
\item Goal 1
\item Goal 2
\item Goal 3
\end{itemize}
\end{gram}

\section{Course Calendar}

You can find a course calendar on Diderot.
%
The course calendar will include times for lectures and office hours.
                
\section{Office Hours}

Using office hour Queue?
\href{https://cmu.ohqueue.com}{CMU Office Houre Queue} 
as a queueing system for TA office hours. 
%
Special instructions: 
When you get on the queue, please include a Zoom link so that the TAs know where to find you. 
%
Office hours will only be held on Zoom and will be posted on Diderot.

\section{Schedule}


The course schedule is included on  Diderot. 
%
Please note that we may make adjustments to the schedule as the
semester progresses.

\section{Recitations}
\label{sec:recitations}

\begin{gram}
[Describe the importance of recitation.]

Current Recitation Sections:
\begin{itemize} 
\item Recitation A: Day, HH:MM - HH:MM pm in Building, Room
\item Recitation B: Day, HH:MM - HH:MM pm in Building, Room
\item Recitation C: Day, HH:MM - HH:MM pm in Building, Room
\end{itemize}
\end{gram}

\section{Exams}


The course includes X midterm exams and a final exam.
%
The midterms will be during the scheduled class period on DATES IF AVAILABLE.
%
The final exam date is to-be-determined. 
%
Plan any travel around exams, as exams will only be rescheduled for timezone issues. 
%
Please notify the professors \textbf{at least 1 week in advance} to reschedule an exam.

\section{Homework}

How many programming or written assignments?
%
How will they be submitted?  
%
See the Schedule and the Online/Written and Programming Books for more details about assignments. 
%
\textbf{All assignments are due at HH:MM Pittsburgh Time (ET)}.

\section{Grading Policies}

\begin{gram}[Course Grading]
Grades will be collected and reported on Diderot.
%

Final scores will be composed of:
\begin{itemize}
\item X\% Midterm exams (each)
\item Y\% Final exam
\item Z\% Homework
\end{itemize}
\end{gram}          

\begin{gram}[Letter Grades]
This class is not curved/curved. 
%
SOME INFORMATION ON HOW THESE ARE CALCULATED
\end{gram}


\begin{gram}[Participation Grades]
Participation will be based on polling questions answered on Diderot WITHIN X HOURS of lecture in Pittsburgh. 
%
There will be some flexibility in computing overall participation grades:
          \begin{itemize}
            \item 5\% for 80\% or greater poll participation
            \item 3\% for 70\%
            \item 1\% for 60\%
            \end{itemize}
%
Correctness of in-class polling responses WILL/WILL-NOT be taken into account for participation grades.

SHOULD THE STUDENT LET THE INSTRUCTORS KNOW ABOUT MISSED CLASSES?
\end{gram}

\begin{gram}[Polls and Academic Integrity]
It is against the course academic integrity policy to participate in a poll when you are not participating to the lecture. 
%
Violations of this policy will be reported as an academic integrity violation. Information about academic integrity at CMU may be found at \href{https://www.cmu.edu/academic-integrity}{https://www.cmu.edu/academic-integrity}
\end{gram}

\subsection{Late Policy}

DESCRIBE LATE POLICY

HOW SHOULD THE STUDENT REQUEST EXTENSIONS IF POSSIBLE?

\subsection{ Collaboration Policy}

Collaboration is encouraged/discouraged.


\section{Communication: Diderot and Email}

\begin{gram}
Diderot makes it more efficient for the course staff and instructors
to handle course related communications, also while reasonably
protecting your privacy.
%
We therefore strongly encourage the students to use Diderot over email. 
%
Your questions and feedback will likely receive better attention and
faster response via Diderot than via email.


For all routine help and questions for the course staff, please use Diderot. 

For more general concerns and feedback, e.g., you are unhappy about
something in the course, please reach out to us as quickly as
possible.  You can do so by creating a Feedback post on Diderot.
Feedback posts are anonymous and therefore you can expect a degree of
privacy (note, however, that the instructors can reveal author of all
posts but do so only under very specific circumstances such as
inappropriate or objectionable language).  This is probably the most
efficient method for many concerns.  

\end{gram}

\subsection{Diderot}

\begin{gram}
Please familiarize yourself with Diderot.  

\begin{enumerate}
%
\item Diderot is organized around \emph{atoms}, blocks of text that
  are "actionable."  You can ask questions or more generally make
  posts about atoms, bookmark them, take (private) notes on them, etc.
  To take or see you notes, press on the "pen and pencil" icon on the
  top right corner of a page.

\item The lecture notes are divided into chapters. Each chapter is a
  collection of atoms organized into sections.  Some atoms
  \emph{expand} to reveal more information, such as examples, that
  might be particularly helpful in initial reads of the material.  You
  can expand an atom by clicking on the triple bar on its left corner.
  You can also expand whole chapters by clicking on the triple bar on
  the right corner of the page.

\item You can make different kinds of posts on Diderot, including
  feedback, question, and social.  Please use the right kind of post
  so that important questions can be answered as promptly as possible.
  For example, if you have some feedback about an atom (e.g., typos),
  then use the "feedback" kind.  If you want to share a meme that you
  have made, please use the "social" kind.  For private and public
  questions, use the "Question" kind.

\item Diderot is an educational platform.  Please be kind, respectful,
  and civil.  Any inappropriate or offensive language is not
  acceptable.  Even if an offending post is anonymous, we reserve the
  right to identify the author. As a student, if you encounter
  inappropriate language, then please let us know.
\end{enumerate}
\end{gram}

\subsection{When to email staff}

\begin{gram}
  Please remember that
    this is a large class and refrain from unnecessarily emailing
    course staff.  We recommend that you avoid emailing course staff
    for any of the following reasons:
  
\begin{enumerate}
\item \emph{Waitlists.} Our department administrators make a sincere
  effort to accommodate our entire waitlist as promptly as possible.
  Course staff (including the instructors) do not accommodate special
  requests and do not have additional information about your
  likelihood of being enrolled.  Please be patient and let the
  department administrators work efficiently to enroll as many people
  as possible.

\item \emph{Grades, exam scores etc.} Rather than email course staff
  about your grades, please talk to a course staff member in an office
  hour if you want feedback about how you could have improved your
  score.


\item \emph{Questions.} Please use Diderot for questions about
  assignments and lecture material.
\end{enumerate}

    We will post announcements, clarifications, corrections, hints, etc. to
    Diderot---please check it on a regular basis. 
  \end{gram}
  

\section{Pandemic Policies}


\begin{gram}[A Tartan's Responsibility]
Covid-19 can be a potentially devastating disease.  
%
In the best case, it is harmless; in the more common case, it is
debilitating for weeks and sometimes for months, with, little known,
but possibly difficult consequences for future well being; in some
not-so-rare cases, it is plain deadly.
%
As a Carnegie Mellon Student, we expect that you understand the
responsibility that you shoulder in this pandemic ad 
%
will do your best to prevent the spread of this disease.
%
This is especially important if you are in excellent health and do not
have any of the conditions that make you susceptible to poor outcomes,
because you will be more likely to pass the virus on to somebody who
might not be as lucky.
%

Now, please read on 
\href
{https://www.cmu.edu/coronavirus/students/tartans-responsibility.html}
{A Tartan's Promise}
%
and play your part so that we can get through these difficult days in the best possible way.
%
We are in this together.
\end{gram}

\begin{gram}[Expectations for In-Person Learning]
In-person classes will look very different for the fall 2020 semester. Everyone in the classroom has to wear a facial covering and maintain physical distancing. 
%
In addition, classroom seating configurations will be set up so that students face the front of the classroom, and ingress into and egress will be carefully managed. 

If you come to class, we expect that 
\begin{itemize}
\item you will wear a face covering,

\item you will maintain physical distancing.
\end{itemize}
%

If you do not abide by these rules, please note that we will 
be subject to student conduct proceedings, which may including removal from CMU. 
\end{gram}

\begin{gram}[Connectivity]
Please make sure that your Internet connection and equipment are set up to use Zoom and able to share audio and video during class meetings.
%
Please see
\href
{https://www.cmu.edu/computing/start/students.html}
{this page from Computing Resources for information on the technology you are likely to need.}
%
Let us know if there is a gap in your technology set-up as soon as
possible, and we can see about finding solutions.
\end{gram}


\begin{gram}[Zoom-Use Policy]
In this class, we will be using Zoom for synchronous (same time) sessions.

In this course, being able to see one another helps to facilitate a better learning environment and promote more engaging discussions. Therefore, our default will be to expect students to have their cameras on during lectures and discussions. 
%
However, we understand there may be reasons students would not want to have their cameras on.

[Options/alternatives to consider based on your class size/format/activities:]

During our class meetings, please keep your mic muted unless you are sharing with the class or your breakout group. 
If you have a question or want to answer a question, please use the chat or the “raise hand” feature (available when the participant list is pulled up). I [or a TA or a rotating student who serves as the “voice of the chat”] will be monitoring these channels in order to call on students to contribute.
Our synchronous meetings will involve breakout room discussions, and those will work better if everyone in your small group has their camera turned on. During large group debriefs, you may keep your video off. 
\end{gram}

\begin{gram}[Remote and In-Person Learning]

WILL LECTURES TAKE PLANE ON ZOOM?

HOW ABOUT RECITATIONS?


TRANSITION TO REMOTE:
At any point during the semester, you may choose to participate in the class remotely. If you decide to switch to remote learning for one or more classes, please try to let us know as soon as possible. 

\end{gram}


\subsection{Recording Lectures and Recitations}
All synchronous lectures and at least one online Recitation will be recorded via Zoom so that students in this course (and only students in this course) can watch or re-watch past class sessions. Please note that breakout rooms will not be recorded. We will make the recordings available on Diderot as soon as possible after each class session (usually within 3 hours of the class meeting). \textbf{Please note that you are not allowed to share these recordings.} This is to protect your FERPA rights and those of your fellow students. 


\section{Accommodations for Students with Disabilities}
If you have a disability and have an accommodations letter from the Disability Resources office, we encourage you to discuss your accommodations and needs with us as early in the semester as possible. We will work with you to ensure that accommodations are provided as appropriate. If you suspect that you may have a disability and would benefit from accommodations but are not yet registered with the Office of Disability Resources, we encourage you to visit their \href{https://www.cmu.edu/disability-resources/}{website}.

\section{Statement of Support for Students’ Health \& Well-being}

\begin{gram}
Take care of yourself. Do your best to maintain a healthy lifestyle this semester by eating well, exercising, getting enough sleep, and taking some time to relax. This will help you achieve your goals and cope with stress.


All of us benefit from support during times of struggle. There are many helpful resources available on campus and an important part of the college experience is learning how to ask for help. Asking for support sooner rather than later is almost always helpful.

If you or anyone you know experiences any academic stress, difficult life events, or feelings like anxiety or depression, we strongly encourage you to seek support. Counseling and Psychological Services (CaPS) is here to help: call 412-268-2922 and visit their website at \href{http://www.cmu.edu/counseling}{http://www.cmu.edu/counseling/}. Consider reaching out to a friend, faculty or family member you trust for help getting connected to the support that can help.
\end{gram}
